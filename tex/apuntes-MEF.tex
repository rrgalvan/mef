% !TEX root = apuntes-MEF.tex
% !TEX spellcheck = es

\documentclass[11pt]{article}

%%%%%%%%%%%%%%%%%%%%%%%%%%%%% Using Packages
\usepackage[spanish]{babel}
\usepackage[utf8]{inputenc}
\usepackage[hmargin={2.7cm,2.2cm}, vmargin={2.2cm, 2.2cm}]{geometry}
\usepackage{graphicx}
\usepackage{amssymb}
\usepackage{amsmath}
\usepackage{amsthm}
\usepackage{graphicx}
\usepackage{color}
\usepackage{xspace}
\usepackage{hyperref}
\hypersetup{
  colorlinks   = true, %Colours links instead of ugly boxes
  urlcolor    = blue, %Colour for external hyperlinks
  linkcolor   = blue, %Colour of internal links
  citecolor   = green %Colour of citations
}

% \usepackage{bm} % Bold math
% \usepackage{empheq}  % Enfatizar, añadir marcos, etc a ecuaciones
% \usepackage{mdframed}  % Cuadros que pueden extenderse a varias páginas
% \usepackage{booktabs} % Quality tables
% \usepackage{psfrag}  % Gráficas (plot) en LaTeX
% \usepackage{pgfplots} % Gráficas (plot) en LaTeX usando PGF


%%%%%%%%%%%%%%%%%%%%%%%%%% Page Setting 
% \geometry{a4paper}
\renewcommand{\baselinestretch}{1.05}

%%%%%%%%%%%%%%%%%%%%%%%%%% Teoremas, etc
\theoremstyle{plain}
\newtheorem{theorem}{Teorema}[section]
\newtheorem{lemma}[theorem]{Lema}
\newtheorem{corolary}[theorem]{Corolario}
\newtheorem{remark}[theorem]{Observación}

\theoremstyle{definition}
\newtheorem{definition}[theorem]{Definición}
\newtheorem*{example}{Ejemplo}
\newtheorem*{exercise}{Ejercicio}
\newtheorem*{properties}{Propiedades}

%%%%%%%%%%%%%%%%%%%%%%%%%% Define some useful colors 
\definecolor{ocre}{RGB}{243,102,25}
\definecolor{mygray}{RGB}{243,243,244}
\definecolor{deepGreen}{RGB}{26,111,0}
\definecolor{shallowGreen}{RGB}{235,255,255}
\definecolor{deepBlue}{RGB}{61,124,222}
\definecolor{shallowBlue}{RGB}{235,249,255}

%%%%%%%%%%%%%%%%%%%%%%%%%% Define an orangebox command 
\newcommand\orangebox[1]{\fcolorbox{ocre}{mygray}{\hspace{1em}#1\hspace{1em}}}

%%%%%%%%%%%%%%%%%%%%%%%%%%%%%%% Plotting Settings 
% \usepgfplotslibrary{colorbrewer}
% \pgfplotsset{width=8cm,compat=1.9}

%%%%%%%%%%%%%%%%%%%%%%%%%%%%%%% Párrafos
\setcounter{secnumdepth}{5}
\newcounter{stepnum}[section]
\newcommand{\step}{\bigskip\noindent\textbf{\thesection.\refstepcounter{stepnum}\thestepnum}.\enspace}

\newcommand{\deff}[1]{\textit{\textbf{#1}}}

%%%%%%%%%%%%%%%%%%%%%%%%%% Mis definiciones
\newcommand{\N}{\ensuremath{\mathbb{N}}}
\newcommand{\Z}{\ensuremath{\mathbb{Z}}}
\newcommand{\Q}{\ensuremath{\mathbb{Q}}}
\newcommand{\R}{\ensuremath{\mathbb{R}}}
\newcommand{\C}{\ensuremath{\mathbb{C}}}
\renewcommand{\P}{\ensuremath{\mathbb{P}}}

\newcommand{\xx}{\ensuremath{\mathbf{x}}\xspace}
\newcommand{\uu}{\ensuremath{\mathbf{u}}\xspace}
\newcommand{\vv}{\ensuremath{\mathbf {v}}\xspace}
\newcommand{\pp}{\ensuremath{p}\xspace}
\newcommand{\qq}{\ensuremath{q}\xspace}
\newcommand{\ff}{\ensuremath{\mathbf{f}}\xspace}
\newcommand{\nn}{\ensuremath{\mathbf{n}}\xspace}

\newcommand{\VV}{\ensuremath{\mathbf{V}}\xspace}
\newcommand{\QQ}{\ensuremath{Q}\xspace}
\newcommand{\XX}{\ensuremath{\mathbf{X}}\xspace}
\newcommand{\Dtest}{\ensuremath{\mathcal{D}}\xspace}

\newcommand{\grad}{\ensuremath{\nabla}}
\renewcommand{\div}{\ensuremath{\nabla\cdot}}
\newcommand{\en}{\quad\text{en}\ }
\newcommand{\sobre}{\quad\text{sobre}\ }

\newcommand{\uxx}{\ensuremath{u_{xx}}\xspace}
\newcommand{\uyy}{\ensuremath{u_{yy}}\xspace}

\newcommand{\norm}[2][]{\ensuremath{\|#2\|_{#1}}}


%%%%%%%%%%%%%%%%%%%%%%%%%%%%%%% Title & Author 
\title{Breve Introducción al Método de los Elementos Finitos para Problemas Elípticos}
\author{J. Rafael Rodríguez Galván}

\begin{document}
\maketitle

%----------------------------------------------------
\section{Introducción}
%----------------------------------------------------

\step
Sea  $\Omega\subset\R^d$ un dominio abierto y acotado, en la práctica $d\in\{1,2,3\}$, con frontera $\partial\Omega$ suficientemente regular%
\footnote{Entenderemos que la frontera es regular si es localmente parametrizable por funciones Lipschitzianas, ver por ejemplo~\cite{Ern-Guermond:04}.}.
En el caso general, dividiremos $\partial\Omega$ en dos partes disjuntas donde impondremos respectivamente condiciones de contorno de tipo Dirichlet y Neumann: $\partial\Omega = \Gamma_D\cup\Gamma_N$,  $\Gamma_D\cap\Gamma_N=\emptyset$.  Podría ser $\Gamma_N=\emptyset$, si no se imponen condiciones Neumann (o $\Gamma_D=\emptyset$ si no hay condiciones Dirichlet).

Consideremos el siguiente problema de Poisson:
\begin{equation}
  \left\{
    \begin{aligned}
      -\Delta u &= f \en{\Omega}, \\
      u&=g_D \sobre{\Gamma_D}, \\
      \grad u\cdot\nn & = g_N \sobre{\Gamma_N},
    \end{aligned}
  \right.
  \label{eq:problema.modelo}
\end{equation}
donde $\Delta u = \partial^ 2 u/\partial x^2 + \partial^ 2 u/\partial y^2$ es el operador laplaciano, $\nn$ denota al vector unitario normal exterior sobre $\partial\Omega$  y las funciones $f:\Omega\to\R$, $g_D:\Gamma_D\to\R$, $g_N:\Gamma_N\to\R$ son datos.

\step
Más adelante veremos que, simplemente asumiendo que $f$, $g_D$ y $g_N$ sean funciones de cuadrado integrable, existe una única solución de~\eqref{eq:problema.modelo} en un sentido débil. Y,  si los datos son funciones continuas en sus dominios, ésta es la solución clásica del problema de Poisson-Dirichlet-Neumann anterior, es decir $u\in C^0(\Omega)\cap C^2(\Omega)$ y las igualdades en~\eqref{eq:problema.modelo} se verifican puntualmente, para todo $\xx\in\overline\Omega$.

\step

El objetivo de este documento es la descripción,  de una forma simple pero rigurosa, del \textbf{método de los elementos finitos} para aproximar la solución de este problema. 
A pesar de que este método despliega toda su potencia en la aproximación de problemas bidimensionales%
\footnote{Para una introducción algo más avanzada, yo habría preferidoabordar directamente el caso bidimensional, como algunos autores de la talla de J. Sayas~\cite{sayas2008gentle}.}, %
comenzaremos estudiando el caso $d=1$ en la siguiente sección. Más adelante se comprobará cómo los razonamientos son fácilmente generalizables  a $d\geq 2$.

La idea de este método consiste en descomponer el dominio como unión de «elementos» más sencillos (por ejemplo, triángulos en el caso $2$--dimensional o en subintervalos en el caso $1$--dimensional) y aproximar la solución de~\eqref{eq:problema.modelo} mediante funciones continuas que, en cada elemento, son polinomios de un grado, $k$, preestablecido. 

En el caso $1$--dimensional y para $k=1$, éstas serán simplemente funciones poligonales. Pero, estas funciones, no son derivables en general. Entonces, ¿en qué sentido podemos decir que verifican~\eqref{eq:problema.modelo}? ¿Qué entendemos por derivada de una función poligonal? La respuesta está en los espacios de funciones derivables en un sentido débil que definiremos en el siguiente apartado.

%----------------------------------------------------
\section{Algunas Nociones de Análisis Funcional}
%----------------------------------------------------
\label{sec:análisis.funcional}

Para las poder definir precisamente las herramientas matemáticas que son objeto de este documento, es conveniente repasar previamente una serie de conceptos básicos sobre espacios de funciones. Los resultados se enuncian sin demostración. Para ver más detalles, pueden consultarse por ejemplo~\cite{adams2003sobolev, sayas2019variational}. En lo que sigue, $\Omega$ denotará a un abierto acotado de $\R^d$, aunque muchos de los conceptos pueden extenderse a dominios no acotados.

%- - - - - - - - - - - - - - - - - - - - - - - - - 
\subsection*{Normas y Productos Escalares}
%- - - - - - - - - - - - - - - - - - - - - - - - - 
Sea $V$ un espacio vectorial sobre el cuerpo\footnote{Por supuesto, todos estos conceptos se generalizan a otros cuerpos como $\Q$} $\R$, finito o infinito dimensional, cuyos elementos estarán formados por funciones. Por ejemplo, $V=\P_2[x]$, el conjunto de los polinomios con coeficientes reales de grado menor o igual a $2$, o $V=C(0,1)$ (funciones continuas en el intervalo $(0,1)$). 

\step 
Un \deff{espacio normado} es cualquier  espacio vectorial $V$ dotado de una \deff{norma}, es decir de una aplicación $\norm[V]{\cdot}$: $v\in V \longmapsto \norm[V]{v}\in\R_+$ tal que: (1) $\norm[V]{v}=0 \Leftrightarrow v=0$, (2) $\norm[V]{\lambda \cdot v}=|\lambda|\cdot\norm[V]{v}$, para todos $\lambda\in\R$, $v\in V$ y (3) $\norm[V]{v+w}\le \norm[V]{v}+\norm[V]{w}$  para todos $v, w\in V$.

\step Un \deff{producto escalar} sobre $V$ es una aplicación bilineal (lineal en sus dos variables),
\[
  (\cdot,\cdot)_V: (v,w)\in V\times V \longmapsto (v,w)_V\in\R
\]
tal que, para todos $v,w\in V$: (1) $(v,w)_V=(w,v)_V$  (2) $(v,v)\ge 0$ y (3) $(v,v)=0 \Leftrightarrow v=0$.

\step 
% \begin{proposition}
  Si $(\cdot,\cdot)_V$ es un producto escalar sobre $V$, la siguiente expresión define una norma en $V$ (\deff{norma subordinada} al producto escalar) 
  \[
    \norm[V]{v} = \big(v,v)_V^{1/2}
  \]
y además se verifica la desigualdad de \textit{\textbf{Cauchy-Schwarz}}:
  \begin{equation}
    (v,w)_V \le \norm[V]{v}\norm[V]{w}, \quad \forall u, v \in V.
  \end{equation}
% \end{proposition}

\step 
Se llama \deff{espacio de Banach} a todo espacio normado $V$ que es completo, es decir, tal que toda sucesión de Cauchy es convergente (para $\norm[V]{\cdot}$) y su límite está en $V$.

\step 
Dados dos espacios normados, $V$ y $W$, se denota por $\mathcal{L}(V,W)$ al conjunto de todas las funcines lineales de $V$ en $W$. Este conjunto es un espacio vectorial y a sus elementos, $A:V\to W$, se les llama \deff{operadores} (lineales).

\step 
Se denomina \deff{espacio de Hilbert} a todo aquel espacio vectorial $V$ dotado de un producto escalar y tal que $V$ es completo para la norma subordinada (por tanto, todo espacio de Hilbert es un espacio de Banach). 
%- - - - - - - - - - - - - - - - - - - - - - - - - 
\subsection*{Espacios de Funciones Lebesgue-Integrables}
%- - - - - - - - - - - - - - - - - - - - - - - - - 

\step 
Denotamos por $L^1(\Omega)$ al espacio de funciones de $\Omega$ en $\R$ que son integrables en el sentido de Lebesgue. Y más en general, dado $p\in [1,+\infty)$, denotamos
\begin{equation*}
  L^p(\Omega) = \big\{ f:\Omega \to \R \ /\ \norm[L^p(\Omega)]{f} < +\infty \big\},
\end{equation*}
donde se define la norma
\begin{equation*}
  \norm[L^p(\Omega)]{f} = \left(\int_\Omega |f|^p\right)^{1/p}.
\end{equation*}

\step
En el caso $p=+\infty$, definimos
\begin{equation*}
  L^\infty(\Omega) = \big\{ f:\Omega \to \R \ /\ \norm[L^\infty(\Omega)]{f} < +\infty \big\},
\end{equation*}
donde
\begin{equation*}
  \norm[L^p(\Omega)]{f} = \mathop{\rm ess\hspace{0.4ex}sup}(f) = \mathop{\rm inf} \{M\in\R+ \ /\ |f(\xx)| \le M \ \text{c.p.t}\ \xx\in\Omega \}.
\end{equation*}

\step 
En el caso $p=2$, $L^2(\Omega)$ es un espacio de Hilbert para el siguiente producto escalar:
\begin{equation}
  \label{producto.escalar.L2}
  (f,g) = (f,g)_{L^2(\Omega)} = \int_\Omega f\cdot g,
\end{equation}
cuya norma asociada es
\[
  \norm[L^2(\Omega)]{f} = \left(\int_\Omega |f|^2\right)^{1/2}.
\]

\step
Algunas propiedades:
\begin{itemize}
  \item 
Si $f^2$ es integrable Riemmann, entonces $f\in L^2(\Omega)$. En particular, toda función continua y acotada en $\Omega$ es integrable y en este sentido $C^(\overline\Omega)\subset L^2(\Omega)$.
  \item El conjunto $L^p(\Omega)$ se generaliza a funciones con valores en $\R^d$, sin más que considerar en~\eqref{producto.escalar.L2} el producto escalar en $\R^d$ (al multiplicar $f\cdot g$).
\end{itemize}

%- - - - - - - - - - - - - - - - - - - - - - - - - 
\subsection*{Espacios de Sobolev}
%- - - - - - - - - - - - - - - - - - - - - - - - - 
\step 
Definimos el \deff{soporte} de una fulnción $\varphi:\Omega\to\R$ como el conjunto
\[
  \mathop{\rm sop}(\varphi) = \overline{\{\xx\in\Omega \ /\ \varphi(\xx)\neq 0\}},
\]
donde $\overline A$ denota a la clausura de un conjunto $A$ cualquiera en $\R$. 

\step
Denotamos por $C^\infty_0(\Omega)$ al espacio de funciones $C^\infty(\Omega)$ cuyo soporte es compacto y está contenido en $\Omega$.

\step
Sea $u\in L^2(\Omega)$. Decimos que $u$ es \deff{derivable en sentido débil}  (con respecto a la variale $x_i$) si existe alguna función $v\in L^2(\Omega)$ tal que
\[
  \int_\Omega v \varphi(x)\; = -\int_\Omega u \partial_{x_i} \varphi \quad \forall \phi \in C^\infty_0(\Omega).
\]
En este caso, decimos que $v$ es la \deff{derivada débil} de u y la denotamos como a la derivada convencional, $v=\partial_{x_i}u$ (o $v=u_{x_i}$ o $v=\partial u/\partial{x_i}$ o $v=D_i u$).

Operadores usuales como el vector \deff{gradiente} o el operador \deff{divergencia} se generalizan a derivadas débiles. La idea de derivada débil se generaliza al concepto más elebante de de distribuciones~\cite{adams2003sobolev}.

\begin{properties}%
\footnote{Ver, por ejemplo,~\cite{adams2003sobolev}.}
\begin{enumerate}
  \item La derivada débil es única salvo conjunto de medida nula.
  \item Si una función es derivable en sentido clásico, entonces existe su derivada débil y coincide (c.p.d.) con la clásica. Basta integrar por partes para comprobarlo.
  \item  Las reglas clásicas para las derivadas de sumas y productos de funciones también son válidas para la derivada débil
\end{enumerate}
\end{properties}
Existen funciones que son derivables en sentido débil pero no en sentido clásico. Por ejemplo, si $u(x)=|x|$ en $\Omega=(-1,1)\subset\R$, entonces 
    \[
      \partial_x u =u'(x)=  \left\{\begin{aligned}-1 & \quad\text{si } x\le 0,\\ 1 &\quad\text{si } x>0. \end{aligned}\right.
    \]

Nota: se puede demostrar que esta última función no (por ser discontinua y $d=1$) no es derivable en sentido débil. Por tanto no existe una «segunda derivada débil» de $u(x)=|x|$.

\step
Definimos el \deff{espacio de Sobolev} $H^1(\Omega)$ como el conjunto de las funciones que tienen todas sus derivadas débiles en $L^2(\Omega)$, esto es
\[
  H^1(\Omega) = \big\{v\in L^2(\Omega) \ /\ \grad v \in [L^2(\Omega)]^d \big\}.
\]

\step 
$H^1(\Omega)$ es un espacio de Hilbert (y por tanto un espacio norrmado completo) para el producto escalar
\[
  (u,v)_{H^1(\Omega)} = (u,v)_{L^2(\Omega)} + (\grad u, \grad v)_{L^2(\Omega)}.
\]

\step Puesto que las funciones de $C^1(\overline\Omega)$ y sus derivadas son de $L^2(\Omega)$,   
\[
C^1(\overline\Omega) \subset H^1(\Omega).
\]

\step 
Consideremos la siguiente función, a la que llamamos \deff{función traza} sobre $\partial\Omega$:
\[
  \gamma:C^0(\overline\Omega) \to C^0(\partial\Omega),
\]
que asocia a cada función de $u\in C^0(\overline\Omega)$ su restricción a $\partial\Omega$, $\gamma(u)=u|_{\partial\Omega}$.
.
El siguiente resultado, una de las claves en la teoría de espacios de Sobolev, garantiza que se puede definir también una la restricción a la frontera
%%
\footnote{Obsérvese que la frontera es un conjunto de medida nula, por tanto no tendría sentido el definir la restricción a $\partial\Omega$ de funciones medibles Lebesgue (y por tanto definida c.p.t. $\xx\in\Omega$).\par 
Por otra parte, es necesario dotar de sentido preciso al espacio $L^2(\partial\Omega)$, ver por ejemplo~\cite{sayas2019variational}.}
%%
de funciones de $H^1(\Omega)$. 
En concreto, se puede demostrar (ver por ejemplo~\cite{sayas2019variational} el siguiente teorema: existe una función lineal y continua, a la que seguiremos llamando traza y denotando por $\gamma$,
\[
  \gamma:  H^1(\Omega) \to L^2(\partial\Omega),
\]
que para funciones continuas coincide con la restricción a $\partial\Omega$, es decir verifica:
\[
  \gamma(u) = u|_{\partial\Omega} \quad  \forall u\in C^0(\overline\Omega)\cap H^1(\Omega).
\]
Abusando del lenguaje, diremos que $u|_{\partial\Omega}=g$ si $\gamma(u)=g$.

\step
Denotamos por $H_0^1(\Omega)$ al núcleo de $\gamma$. Es decir:
\[
  H_0^1(\Omega) = \{ u\in H^1(\Omega) \ /\ \gamma(u)=0\}.
\]

\step Se verifica la siguiente propiedad: $C_0^\infty(\Omega)\subset H_0^1(\Omega)$ y de hecho%
%
\footnote{El hecho de que los espacios de Sobolev pueden ser definidos por completitud no es evidente. Este fue el objeto de una curiosa publicación de sólo dos páginas con el intenioso título «$H=W$»~\cite{meyers1964h}.}
$$H_0^1(\Omega) = \overline{C_0^\infty(\Omega)}^{H^1(\Omega)},$$ clausura respecto a la norma de $H^1(\Omega)$.

\step \textbf{Desigualdad de Poincaré-Friecrichs}: si $\Omega$ es acotado\footnote{Basta acotado en alguna dirección}, existe una constante $C>0$ tal que 
\[
  \norm[\Omega]{u} \le C \norm[\Omega]{\grad u}, \quad\forall u\in H_0^1(\Omega).
\]

\step Consecuencia: la siguiente expresión define una norma en $H_0^1(\Omega)$ que, de hecho, es equivalente en este espacio a la norma de $H^1(\Omega)$:
\[
  \norm[H_0^1]{u} = \left(\int_{\Omega} \norm{\grad u}^2 \right)^{1/2}.
\]

%----------------------------------------------------
\section{El Problema de Poisson--Dirichlet Unidimensional}
%----------------------------------------------------

Comenzaremos considerando particular de~\eqref{eq:problema.modelo} en el que $d=1$, siendo $\Omega=(a,b)\subset\R$, $a,b\in\R$, y con condiciones de contorno Dirichlet homogéneas. Es decir, $\Gamma_D=\{a,b\}$ y $\Gamma_N=\emptyset$. Tenemos así el siguiente problema de Poisson-Dirichlet: 
\begin{equation}
  \left\{
    \begin{aligned}
      - \uxx &= f \en{\Omega}, \\
      u&=0 \sobre{\Gamma_D}, \\
    \end{aligned}
  \right.
  \label{eq:problema.dirichlet.homogeneo.1d}
\end{equation}
siendo $\uxx = d^2 u/d x^2$ y $f\in C^0(\Omega)$. Una solución clásica será una función $u\in C^(\overline\Omega)\cap C^1(\Omega)$ para las que se verifica~\eqref{eq:problema.dirichlet.homogeneo.1d} de forma puntual, es decir $-u_{xx}(x)=f(x)$ para todo $x\in(a,b)$ y $u(a)=u(b)=0$.


%----------------------------------------------------
\section{Formulación Variacional}
%----------------------------------------------------

En este apartado se estudia una formulación alternativa para el problema~\eqref{eq:problema.dirichlet.homogeneo.1d}. Veremos que ésta permite establecer condiciones suficientes para la existencia y unicidad de solución, en un sentido débil o más general que el de solución clásica. Estas soluciones débiles serán, de hecho, soluciones clásicas siempre que los datos sean suficientemente regulares.

Como veremos luego, este tipo de formulación débil o variacional es generalizable de forma sencilla a $d>1$, así como a  condiciones de contorno más generales. Y será la base sobre la que  se construye el método de los elementos finitos. 


\step
Suponiendo que $u$ es una solución clásica del problema~\eqref{eq:problema.dirichlet.homogeneo.1d} y $v$ es una función de $C^1(\overline\Omega)$ tal que $v=0$ sobe $\Gamma_D$, basta integrar por partes para obtener:
\begin{equation*}
  \int_\Omega (-\uxx)\,v  = -  \int_a^b \uxx(x) v(x) \,dx = 
  \int_a^b u_x(x) v_x(x)\,dx - \big( u(b)v(b) - u(a)v(a)  \big).
\end{equation*}
Y usando que $v(a)=v(b)=0$ y que $-\uxx(x)=f(x)$ para todo $x\in(a,b)$:
\[
  \int_a^b u_x(x) v_x(x)\,dx  = \int_a^b f(x) v(x) \, dx.
\]
Es decir, toda solución clásica $u$ de~\eqref{eq:problema.dirichlet.homogeneo.1d} verifica, para toda $v\in C^1(\overline\Omega)$ tal que $v(a)=v(b)=0$, una ecuación de la forma:
\begin{equation}
  a(u,v) = (f,v),
\end{equation}
donde denotamos
\begin{align}
  a(u,v)  &= \int_a^b u_x(x) v_x(x)\,dx,  
  \label{eq:forma.bilineal.a}
\\
  (f,v)  &= \int_a^b f(x) v(x) \, dx.
  \label{eq:producto.l2}
\end{align}

\step
Para que la última integral esté bien definida, no es necesario que $f$ y $u$ sean funciones de $C^0([a,b])$: es suficiente que $f$ y $v$ sean funciones de $L^2(\Omega)$, el espacio vectorial de funciones de cuadrado integrable. 
De hecho, la expresión~\eqref{eq:producto.l2} es un producto escalar en $L^2(\Omega)$. 

Y para que la integral~\eqref{eq:forma.bilineal.a} esté bien definida, no es necesario que $u$ y $v$ sean funciones de $C^1(\Omega)$. Bastaría que tanto $u$ como $v$ tengan derivadas $u'$ y $v'$ en un sentido débil (ver apéndice~\ref{sec:definiciones.basicas}) y que $u',v'\in L^2(\Omega)$. Precisamente, al espacio de funciones de $L^2(\Omega)$ cuya derivada es también de $L^2(\Omega)$ lo denotamos\footnote{
  De nuevo, y aunque sea menos evidente, la expresión~\eqref{eq:forma.bilineal.a} es un producto escalar en $H^1(\Omega)$.
} por $H^1(\Omega)$ y así podemos decir que~\eqref{eq:forma.bilineal.a} está bien definido para todos $u,v\in H^1(\Omega)$.



%
% \step
% Como se detalla en el apéndice~\ref{sec:definiciones.basicas}, la expresión $(\cdot,\cdot)$ no es más que el producto escalar en $L^2(\Omega)$ (el espacio de funciones de cuadrado integrable). Por tanto,
% \[
%   
% \]
%
% Y a su vez $a(\cdot,\cdot)$ es una forma bilineal en el espacio $H^1(\Omega)$ (el espacio de funciones de $L^2(\Omega)$ con derivadas débiles que son también funciones de $L^2(\Omega)$).
%
Por tanto, si definimos el espacio
\begin{equation*}
  V= H_0^1(\Omega) = \{v\in H^1(\Omega), \ /\ v(a)=v(b)=0\},
\end{equation*}
podemos decir que toda solución clásica $u$ de~\eqref{eq:problema.dirichlet.homogeneo.1d} verifica la siguiente propiedad:
\begin{equation}
u \in V, \quad a(u,v) = (f,v) \quad \forall v\in V.
  \label{eq:formulacon.variacional.1d}
\end{equation}



\bibliographystyle{unsrt}
\bibliography{biblio}
\end{document}


