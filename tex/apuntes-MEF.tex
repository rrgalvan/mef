\documentclass[11pt,leqno]{article}

%%%%%%%%%%%%%%%%%%%%%%%%%%%%% Using Packages
\usepackage[spanish]{babel}
\usepackage[utf8]{inputenc}
\usepackage[margin={2.5cm,2cm}]{geometry}
\usepackage{graphicx}
\usepackage{amssymb}
\usepackage{amsmath}
\usepackage{amsthm}
\usepackage{graphicx}
\usepackage{color}
\usepackage{xspace}
% \usepackage{bm} % Bold math
% \usepackage{empheq}  % Enfatizar, añadir marcos, etc a ecuaciones
% \usepackage{mdframed}  % Cuadros que pueden extenderse a varias páginas
% \usepackage{booktabs} % Quality tables
% \usepackage{psfrag}  % Gráficas (plot) en LaTeX
% \usepackage{pgfplots} % Gráficas (plot) en LaTeX usando PGF


%%%%%%%%%%%%%%%%%%%%%%%%%% Page Setting 
% \geometry{a4paper}
\renewcommand{\baselinestretch}{1.05}

%%%%%%%%%%%%%%%%%%%%%%%%%% Teoremas, etc
\theoremstyle{plain}
\newtheorem{theorem}{Teorema}[section]
\newtheorem{lemma}[theorem]{Lema}
\newtheorem{corolary}[theorem]{Corolario}
\newtheorem{remark}[theorem]{Observación}

\theoremstyle{definition}
\newtheorem{definition}[theorem]{Definición}
\newtheorem*{example}{Ejemplo}
\newtheorem*{exercise}{Ejercicio}


%%%%%%%%%%%%%%%%%%%%%%%%%% Define some useful colors 
\definecolor{ocre}{RGB}{243,102,25}
\definecolor{mygray}{RGB}{243,243,244}
\definecolor{deepGreen}{RGB}{26,111,0}
\definecolor{shallowGreen}{RGB}{235,255,255}
\definecolor{deepBlue}{RGB}{61,124,222}
\definecolor{shallowBlue}{RGB}{235,249,255}

%%%%%%%%%%%%%%%%%%%%%%%%%% Define an orangebox command 
\newcommand\orangebox[1]{\fcolorbox{ocre}{mygray}{\hspace{1em}#1\hspace{1em}}}

%%%%%%%%%%%%%%%%%%%%%%%%%%%%%%% Plotting Settings 
% \usepgfplotslibrary{colorbrewer}
% \pgfplotsset{width=8cm,compat=1.9}

%%%%%%%%%%%%%%%%%%%%%%%%%%%%%%% Párrafos
\setcounter{secnumdepth}{5}
\newcounter{stepnum}[section]
\newcommand{\step}{\bigskip\noindent\textbf{\thesection.\refstepcounter{stepnum}\thestepnum}.\enspace}

%%%%%%%%%%%%%%%%%%%%%%%%%% Mis definiciones
\newcommand{\R}{\ensuremath{\mathbb{R}}}
\newcommand{\Rset}{\ensuremath{\mathbb{R}}}
\newcommand{\xx}{\ensuremath{\mathbf{x}}\xspace}
\newcommand{\uu}{\ensuremath{\mathbf{u}}\xspace}
\newcommand{\vv}{\ensuremath{\mathbf {v}}\xspace}
\newcommand{\pp}{\ensuremath{p}\xspace}
\newcommand{\qq}{\ensuremath{q}\xspace}
\newcommand{\ff}{\ensuremath{\mathbf{f}}\xspace}
\newcommand{\nn}{\ensuremath{\mathbf{n}}\xspace}

\newcommand{\VV}{\ensuremath{\mathbf{V}}\xspace}
\newcommand{\QQ}{\ensuremath{Q}\xspace}
\newcommand{\XX}{\ensuremath{\mathbf{X}}\xspace}
\newcommand{\Dtest}{\ensuremath{\mathcal{D}}\xspace}

\newcommand{\grad}{\ensuremath{\nabla}}
\renewcommand{\div}{\ensuremath{\nabla\cdot}}
\newcommand{\en}{\quad\text{en}\ }
\newcommand{\sobre}{\quad\text{sobre}\ }

\newcommand{\norm}[2][]{\ensuremath{\|#2\|_{#1}}}


%%%%%%%%%%%%%%%%%%%%%%%%%%%%%%% Title & Author 
\title{El Método de los Elementos Finitos para Problemas Elípticos}
\author{J. Rafael Rodríguez Galván}

\begin{document}
\maketitle

%-------------------------------------------------------------
\section{Introducción}
%-------------------------------------------------------------

El objetivo de este documento es la descripción del método de los elementos finitos para problemas elípticos de una forma simple pero rigurosa. 

\step
Sea  $\Omega\subset\Rset^d$ un dominio abierto y acotado, donde en la práctica $d\in\{1,2,3\}$, con frontera $\partial\Omega$ suficientemente regular%
\footnote{Entenderemos que la frontera es regular si es localmente parametrizable por funciones Lipschitzianas, ver por ejemplo~\cite{Ern-Guermond:04}}.
En el caso general, dividiremos $\partial\Omega$ en dos partes disjuntas donde impondremos respectivamente condiciones de contorno de tipo Dirichlet y Neumann: $\partial\Omega = \Gamma_D\cup\Gamma_N$,  $\Gamma_D\cap\Gamma_N=\emptyset$.  Podría ser $\Gamma_N=\emptyset$, si no se imponen condiciones Neumann (o $\Gamma_D=\emptyset$ si no hay condiciones Dirichlet).

Consideremos el siguiente problema de Poisson:
\begin{equation}
  \left\{
    \begin{aligned}
      -\Delta u &= f \en{\Omega}, \\
      u&=g_D \sobre{\Gamma_D}, \\
      \grad u\cdot\nn & = g_N \sobre{\Gamma_N},
    \end{aligned}
  \right.
  \label{eq:problema.modelo}
\end{equation}
donde $\Delta u = \partial^ 2 u/\partial x^2 + \partial^ 2 u/\partial y^2$ es el operador laplaciano, $\nn$ denota al vector unitario normal exterior sobre $\partial\Omega$  y las funciones $f:\Omega\to\R$, $g_D:\Gamma_D\to\R$, $g_N:\Gamma_N\to\R$ son datos.

\step
Más adelante veremos que, simplemente asumiendo que $f$, $g_D$ y $g_N$ sean funciones de cuadrado integrable, existe una única solución de~\eqref{eq:problema.modelo} en un sentido débil. Y,  si los datos son funciones continuas en sus dominios, ésta es la solución clásica del problema de Poisson-Dirichlet-Neumann anterior, es decir $u\in C^0(\Omega)\cap C^2(\Omega)$ y las igualdades en~\eqref{eq:problema.modelo} se verifican puntualmente, para todo $\xx\in\overline\Omega$.

A pesar de que este método despliega toda su potencia en la aproximación de problemas bidimensionales%
\footnote{Para una introducción algo más avanzada, yo habría preferidoabordar directamente el caso bidimensional, como algunos autores de la talla de J. Sayas~\cite{sayas2008gentle}}, %
comenzaremos estudiando el caso $d=1$ en la siguiente sección. 

%-------------------------------------------------------------
\section{El Problema de Poisson en Dominios Unidimensionales}
%-------------------------------------------------------------

\bibliographystyle{unsrt}
\bibliography{biblio}
\end{document}
