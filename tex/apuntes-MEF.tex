% !TEX root = apuntes-MEF.tex
% !TEX spellcheck = es

\documentclass[11pt]{article}

%%%%%%%%%%%%%%%%%%%%%%%%%%%%% Using Packages
\usepackage[spanish]{babel}
\usepackage[utf8]{inputenc}
\usepackage[hmargin={2.7cm,2.2cm}, vmargin={2.2cm, 2.2cm}]{geometry}
\usepackage{graphicx}
\usepackage{amssymb}
\usepackage{amsmath}
\usepackage{amsthm}
\usepackage{graphicx}
\usepackage{color}
\usepackage{xspace}
\usepackage{hyperref}
\hypersetup{
  colorlinks   = true, %Colours links instead of ugly boxes
  urlcolor    = blue, %Colour for external hyperlinks
  linkcolor   = blue, %Colour of internal links
  citecolor   = green %Colour of citations
}

% \usepackage{bm} % Bold math
% \usepackage{empheq}  % Enfatizar, añadir marcos, etc a ecuaciones
% \usepackage{mdframed}  % Cuadros que pueden extenderse a varias páginas
% \usepackage{booktabs} % Quality tables
% \usepackage{psfrag}  % Gráficas (plot) en LaTeX
% \usepackage{pgfplots} % Gráficas (plot) en LaTeX usando PGF

% Cambiar el formato de la fecha, para usarlo como número de vesión
\usepackage[yyyymmdd]{datetime}
\renewcommand{\dateseparator}{.}
\newdateformat{versiondate}{%
Versión: \twodigit{\THEMONTH}.\twodigit{\THEDAY}\enspace(\twodigit{\THEYEAR})}

%%%%%%%%%%%%%%%%%%%%%%%%%% Page Setting 
% \geometry{a4paper}
\renewcommand{\baselinestretch}{1.05}

%%%%%%%%%%%%%%%%%%%%%%%%%% Teoremas, etc
\theoremstyle{plain}
\newtheorem{theorem}{Teorema}[section]
\newtheorem{lemma}[theorem]{Lema}
\newtheorem{corolary}[theorem]{Corolario}
\newtheorem{remark}[theorem]{Observación}

\theoremstyle{definition}
\newtheorem{definition}[theorem]{Definición}
\newtheorem*{example}{Ejemplo}
\newtheorem*{exercise}{Ejercicio}
\newtheorem*{properties}{Propiedades}

%%%%%%%%%%%%%%%%%%%%%%%%%% Define some useful colors 
\definecolor{ocre}{RGB}{243,102,25}
\definecolor{mygray}{RGB}{243,243,244}
\definecolor{deepGreen}{RGB}{26,111,0}
\definecolor{shallowGreen}{RGB}{235,255,255}
\definecolor{deepBlue}{RGB}{61,124,222}
\definecolor{shallowBlue}{RGB}{235,249,255}


%%%%%%%%%%%%%%%%%%%%%%%%%%%%%%% Plotting Settings 
% \usepgfplotslibrary{colorbrewer}
% \pgfplotsset{width=8cm,compat=1.9}

%%%%%%%%%%%%%%%%%%%%%%%%%%%%%%% Párrafos
\setcounter{secnumdepth}{5}
\newcounter{stepnum}[section]
\newcommand{\step}[1][]{\bigskip\noindent\textbf{\thesection.\refstepcounter{stepnum}\thestepnum}.\enspace{#1}}

\renewcommand{\theparagraph}{\arabic{section}.\arabic{paragraph}}
\renewcommand{\step}[1][]{\paragraph{#1}\hspace{-1.1em}}

\newcommand{\deff}[1]{\textit{\textbf{#1}}}

%%%%%%%%%%%%%%%%%%%%%%%%%% Mis definiciones
\newcommand{\N}{\ensuremath{\mathbb{N}}}
\newcommand{\Z}{\ensuremath{\mathbb{Z}}}
\newcommand{\Q}{\ensuremath{\mathbb{Q}}}
\newcommand{\R}{\ensuremath{\mathbb{R}}}
\newcommand{\C}{\ensuremath{\mathbb{C}}}

\newcommand{\xx}{\ensuremath{\mathbf{x}}\xspace}
\newcommand{\uu}{\ensuremath{\mathbf{u}}\xspace}
\newcommand{\vv}{\ensuremath{\mathbf {v}}\xspace}
\newcommand{\pp}{\ensuremath{p}\xspace}
\newcommand{\qq}{\ensuremath{q}\xspace}
\newcommand{\ff}{\ensuremath{\mathbf{f}}\xspace}
\newcommand{\nn}{\ensuremath{\mathbf{n}}\xspace}

\newcommand{\VV}{\ensuremath{\mathbf{V}}\xspace}
\newcommand{\QQ}{\ensuremath{Q}\xspace}
\newcommand{\XX}{\ensuremath{\mathbf{X}}\xspace}
\newcommand{\Dtest}{\ensuremath{\mathcal{D}}\xspace}

\newcommand{\grad}{\ensuremath{\nabla}}
\renewcommand{\div}{\ensuremath{\nabla\cdot}}
\newcommand{\en}{\quad\text{en}\ }
\newcommand{\sobre}{\quad\text{sobre}\ }

\newcommand{\uxx}{\ensuremath{u_{xx}}\xspace}
\newcommand{\uyy}{\ensuremath{u_{yy}}\xspace}

\newcommand{\norm}[2][]{\ensuremath{\|#2\|_{#1}}}

\renewcommand{\P}{\ensuremath{\mathbb{P}}}
\newcommand{\TestSpace}{\ensuremath{\mathcal{D}}}

%%%%%%%%%%%%%%%%%%%%%%%%%%%%%%% Title & Author 
\title{Breve Introducción al Método de los Elementos Finitos para Problemas Elípticos}
\author{J. Rafael Rodríguez Galván}
\date{\versiondate\today}

\begin{document}
\maketitle

%----------------------------------------------------
\section{Introducción}
%----------------------------------------------------

\step
Sea  $\Omega\subset\R^d$ un dominio abierto y acotado, en la práctica $d\in\{1,2,3\}$, con frontera $\partial\Omega$ suficientemente regular%
\footnote{Entenderemos que la frontera es regular si es localmente parametrizable por funciones Lipschitzianas, ver por ejemplo~\cite{Ern-Guermond:04}.}.
En el caso general, dividiremos $\partial\Omega$ en dos partes disjuntas donde impondremos respectivamente condiciones de contorno de tipo Dirichlet y Neumann: $\partial\Omega = \Gamma_D\cup\Gamma_N$,  $\Gamma_D\cap\Gamma_N=\emptyset$.  Podría ser $\Gamma_N=\emptyset$, si no se imponen condiciones Neumann (o $\Gamma_D=\emptyset$ si no hay condiciones Dirichlet).

Consideremos el siguiente problema de Poisson:
\begin{equation}
  \left\{
    \begin{aligned}
      -\Delta u &= f \en{\Omega}, \\
      u&=g_D \sobre{\Gamma_D}, \\
      \grad u\cdot\nn & = g_N \sobre{\Gamma_N},
    \end{aligned}
  \right.
  \label{eq:problema.modelo}
\end{equation}
donde $\Delta u = \partial^ 2 u/\partial x^2 + \partial^ 2 u/\partial y^2$ es el operador laplaciano, $\nn$ denota al vector unitario normal exterior sobre $\partial\Omega$  y las funciones $f:\Omega\to\R$, $g_D:\Gamma_D\to\R$, $g_N:\Gamma_N\to\R$ son datos.

\step
Más adelante veremos que, simplemente asumiendo que $f$, $g_D$ y $g_N$ sean funciones de cuadrado integrable, existe una única solución de~\eqref{eq:problema.modelo} en un sentido débil. Y,  si los datos son funciones continuas en sus dominios, ésta es la solución clásica del problema de Poisson-Dirichlet-Neumann anterior, es decir $u\in C^0(\Omega)\cap C^2(\Omega)$ y las igualdades en~\eqref{eq:problema.modelo} se verifican puntualmente, para todo $\xx\in\overline\Omega$.

\step

El objetivo de este documento es la descripción,  de una forma simple pero rigurosa, del \textbf{método de los elementos finitos} para aproximar la solución de este problema. 
A pesar de que este método despliega toda su potencia en la aproximación de problemas bidimensionales%
\footnote{Para una introducción algo más avanzada, yo habría preferidoabordar directamente el caso bidimensional, como algunos autores de la talla de J. Sayas~\cite{sayas2008gentle}.}, %
comenzaremos estudiando el caso $d=1$ en la siguiente sección. Más adelante se comprobará cómo los razonamientos son fácilmente generalizables  a $d\geq 2$.

La idea de este método consiste en descomponer el dominio como unión de «elementos» más sencillos (por ejemplo, triángulos en el caso $2$--dimensional o en subintervalos en el caso $1$--dimensional) y aproximar la solución de~\eqref{eq:problema.modelo} mediante funciones continuas que, en cada elemento, son polinomios de un grado, $k$, preestablecido. 

En el caso $1$--dimensional y para $k=1$, éstas serán simplemente funciones poligonales. Pero, estas funciones, no son derivables en general. Entonces, ¿en qué sentido podemos decir que verifican~\eqref{eq:problema.modelo}? ¿Qué entendemos por derivada de una función poligonal? La respuesta está en los espacios de funciones derivables en un sentido débil que definiremos en el siguiente apartado.

%----------------------------------------------------
\section{Algunas nociones de análisis funcional}
%----------------------------------------------------
\label{sec:análisis.funcional}

Para las poder definir precisamente las herramientas matemáticas que son objeto de este documento, es conveniente repasar previamente una serie de conceptos básicos sobre espacios de funciones. Los resultados se enuncian sin demostración. Para ver más detalles, pueden consultarse por ejemplo~\cite{adams2003sobolev, sayas2019variational}. En lo que sigue, $\Omega$ denotará a un abierto acotado de $\R^d$, aunque muchos de los conceptos pueden extenderse a dominios no acotados.

%- - - - - - - - - - - - - - - - - - - - - - - - - 
\subsection*{Normas y productos escalares}
%- - - - - - - - - - - - - - - - - - - - - - - - - 
Sea $V$ un espacio vectorial sobre el cuerpo\footnote{Por supuesto, todos estos conceptos se generalizan a otros cuerpos como $\Q$} $\R$, finito o infinito dimensional, cuyos elementos estarán formados por funciones. Por ejemplo, $V=\P_2[x]$, el conjunto de los polinomios con coeficientes reales de grado menor o igual a $2$, o $V=C(0,1)$ (funciones continuas en el intervalo $(0,1)$). Las siguientes afirmaciones generalizan los conceptos básicos en espacios vectoriales  

\step 
Un \deff{espacio normado} es cualquier  espacio vectorial $V$ dotado de una \deff{norma}, es decir de una aplicación $\norm[V]{\cdot}$: $v\in V \longmapsto \norm[V]{v}\in\R_+$ tal que: (1) $\norm[V]{v}=0 \Leftrightarrow v=0$, (2) $\norm[V]{\lambda \cdot v}=|\lambda|\cdot\norm[V]{v}$, para todos $\lambda\in\R$, $v\in V$ y (3) $\norm[V]{v+w}\le \norm[V]{v}+\norm[V]{w}$  para todos $v, w\in V$.

\step Dos \deff{normas equivalentes} en $V$ son todas aquellas $\norm[(1)]{\cdot}$ y  $\norm[(2)]{\cdot}$ tales que existen $C_1, C_2>0$ para las que
\[
  \norm[(2)]{v} \le C_1\norm[(1)]{v} \quad{y}\quad \norm[(1)]{v} \le C_2\norm[(2)]{v}, \quad \forall v\in V.
\]
En espacios finito-dimensionales, todas las normas son equivalentes, pero no en espacios de dimensión infinita.

\step Dados dos espacios normados $V$ y $W$, una aplicación lineal $L:V\to W$, $v\in V\mapsto Lv=L(v)\in W$, es continua si existe $C>0$ tal que $\norm[W]{Lv} \le C \norm[V]{v}$ para todo $v\in V$.

Se denota por $\mathcal{L}(V,W)$ al conjunto de todas las aplicaciones lineales y continuas de $V$ en $W$. Este conjunto es un espacio vectorial y a sus elementos se les llama \deff{operadores} (lineales).

\step Una \deff{forma bilineal} sobre $V\times W$ es una aplicación $a:(v,w)\in V\times W \mapsto a(v,w)\in\R$ que es lineal y continua para cada una de sus dos componentes. Es decir, existe $C>0$ tal que:
\[
  a(v,w) \le C \norm[V]{v}\,\norm[W]{w} \quad\forall (v,w)\in V\times W.
\]

\step Un \deff{producto escalar} sobre $V$ es una forma bilineal 
\[
  (\cdot,\cdot)_V: (v,w)\in V\times V \longmapsto (v,w)_V\in\R
\]
tal que, para todos $v,w\in V$: (1) $(v,w)_V=(w,v)_V$  (2) $(v,v)\ge 0$ y (3) $(v,v)=0 \Leftrightarrow v=0$.


\step 
% \begin{proposition}
  Si $(\cdot,\cdot)_V$ es un producto escalar sobre $V$, la siguiente expresión define una norma en $V$ (\deff{norma subordinada} al producto escalar) 
  \[
    \norm[V]{v} = \big(v,v)_V^{1/2}
  \]
y además se verifica la desigualdad de \textit{\textbf{Cauchy-Schwarz}}:
  \begin{equation}
    (v,w)_V \le \norm[V]{v}\norm[V]{w}, \quad \forall u, v \in V.
  \end{equation}
% \end{proposition}

\step 
Se llama \deff{espacio de Banach} a todo espacio normado $V$ que es completo, es decir, tal que toda sucesión de Cauchy es convergente (para $\norm[V]{\cdot}$) y su límite está en $V$.

\step 
Se denomina \deff{espacio de Hilbert} a todo aquel espacio vectorial $V$ dotado de un producto escalar y tal que $V$ es completo para la norma subordinada (por tanto, todo espacio de Hilbert es un espacio de Banach). 

\step 
Se verifica la siguiente propiedad: todo subespacio vectorial cerrado de un espacio de Hilbert también un espacio de Hilbert para el mismo producto escalar. 

%- - - - - - - - - - - - - - - - - - - - - - - - - 
\subsection*{Espacios de funciones Lebesgue-integrables}
%- - - - - - - - - - - - - - - - - - - - - - - - - 

\step 
Denotamos por $L^1(\Omega)$ al espacio de funciones de $\Omega$ en $\R$ que son integrables en el sentido de Lebesgue%
\footnote{Se prefiere usar integrales de Lebesgue por motivos de completitud, es decir, porque bajo hipóteis razonables el límite de funciones integrables es integrable. Esto no ocurre con la integral de Riemann. Véase~\cite{Brenner-Scott:08}, sección 1.1, para un ejemplo motivador.}
%
. Y más en general, dado $p\in [1,+\infty)$, denotamos
\begin{equation*}
  L^p(\Omega) = \big\{ f:\Omega \to \R \ /\ \norm[L^p(\Omega)]{f} < +\infty \big\},
\end{equation*}
donde se define la norma
\begin{equation*}
  \norm[L^p(\Omega)]{f} = \left(\int_\Omega |f|^p\right)^{1/p}.
\end{equation*}

\step
En el caso $p=+\infty$, definimos
\begin{equation*}
  L^\infty(\Omega) = \big\{ f:\Omega \to \R \ /\ \norm[L^\infty(\Omega)]{f} < +\infty \big\},
\end{equation*}
donde
\begin{equation*}
  \norm[L^p(\Omega)]{f} = \mathop{\rm ess\hspace{0.4ex}sup}(f) = \mathop{\rm inf} \{M\in\R+ \ /\ |f(\xx)| \le M \ \text{c.p.t}\ \xx\in\Omega \}.
\end{equation*}

\step 
En el caso $p=2$, $L^2(\Omega)$ es un espacio de Hilbert para el siguiente producto escalar:
\begin{equation}
  \label{producto.escalar.L2}
  (f,g) = (f,g)_{L^2(\Omega)} = \int_\Omega f\cdot g,
\end{equation}
cuya norma asociada es
\[
  \norm[L^2(\Omega)]{f} = \left(\int_\Omega |f|^2\right)^{1/2}.
\]

\step
Algunas propiedades:
\begin{itemize}
  \item 
Si $f^2$ es integrable Riemmann, entonces $f\in L^2(\Omega)$. En particular, toda función continua y acotada en $\Omega$ es integrable y en este sentido $C^0(\overline\Omega)\subset L^2(\Omega)$.
  \item El conjunto $L^p(\Omega)$ se generaliza a funciones con valores en $\R^d$, sin más que considerar en~\eqref{producto.escalar.L2} el producto escalar en $\R^d$.
\end{itemize}

%- - - - - - - - - - - - - - - - - - - - - - - - - 
\subsection*{Derivadas débiles y espacios de Sobolev}
%- - - - - - - - - - - - - - - - - - - - - - - - - 
El resto de esta sección debe verse como una introducción rápida y superficial a algunos resultados de análisis funcional que no son, en absoluto, evidentes. Con el fin de tratar solamente los conceptos que serán necesarios en las próximas secciones, se han evitado algunos de ellos, como la teoría de distribuciones, que habrían sido más generales y dotado de más rigor a los contenidos. Se recomienda profundizar en esta fascinante teoría matemática, por ejemplo a través de los libros de J. Sayas et al~\cite{sayas2019variational} o el clásico de Adams et al~\cite{adams2003sobolev}.

\step[\textbf{Fórmulas de integración por partes}].
La siguiente identidad se puede demostrar en un sentido clásico y en una amplia familia de dominios $\Omega$ acotados, en particular en dominios con frontera poligonal a trozos como los que utilizaremos al aplicar el método de los elementos finitos. Puede accederse a una demostración, por ejemplo, en~\cite{sayas2019variational}.
Si $f\in C^1(\overline\Omega)$, entonces
\begin{equation}
  \label{eq:partes.1}
  \int_\Omega \partial_{x_i} f(\xx) \, d\xx =
  \int_{\partial\Omega} f(\xx)\,n_i \, dS,
\end{equation}
donde para todo $\xx\in\partial\Omega$, $\nn=\nn(\xx)=(n_1,\dots,n_d)$ es el vector normal unitario orientado al exterior de $\Omega$.
Si $f,g\in C^1(\overline\Omega)$, aplicando~\eqref{eq:partes.1} al producto $f\cdot g$ 
% y usando que $\partial_{x_i} (f\cdot g) = (\partial_{x_i}f)\cdot g+f\cdot(\partial_{x_i}g)$, 
se tiene la fórmula de integración por partes
\begin{equation}
  \label{eq:partes.2}
  \int_\Omega \partial_{x_i}f(\xx)\cdot g(\xx) \, d\xx    =
  - \int_\Omega f(\xx)\cdot \partial_{x_i}g(\xx) \, d\xx    
  +
  \int_{\partial\Omega} f(\xx)g(\xx)\,n_i \, dS.
\end{equation}
% Sea ahora $\ff=(f_1,\dots,f_d)\in C^1(\overline\Omega; \R^d)$ y sea $\div \ff = \partial_{x_1} f_1 + \cdots + \partial_{x_d} f_d$ la \deff{divergencia} de $\ff$. Sumando la ecuación \eqref{eq:partes.1} para $i=1,\dots,d$, se tiene el teorema de la divergencia:
% \[
%   \int_\Omega \div\ff (\xx) \,\dx = 
%   \int_{\partial\Omega} \ff(\xx)\cdot\nn \, dS.
% \]
Aplicando la misma idea para el producto $\partial_{x_i} f \cdot g$, y sumando el resultado para $i=1,\dots d$, se tiene la siguiente fórmula de integración por partes, conocida como \textbf{fórmula de Green}:
\begin{equation}
\label{eq:green}
\int_\Omega \Delta f(\xx)\, g(\xx) \, d\xx =
- \int_\Omega \nabla f(\xx)\, \nabla g(\xx) \, d\xx
+ \int_{\partial\Omega} (\nabla f(\xx)\cdot\nn)\, g(\xx)  \, dS.
\end{equation}
\step
La identidad\eqref{eq:partes.2} implica que la derivada de toda $f\in C^1(\overline\Omega)$ verifica la siguiente propiedad:
\begin{equation}
  \label{eq:partes.3}
  \int_\Omega \partial_{x_i}f(\xx)\cdot \varphi(\xx) \, d\xx    =
  - \int_\Omega f(\xx)\cdot \partial_{x_i}\varphi(\xx) \, d\xx    \quad \forall \varphi\in C^1(\overline\Omega)  \ / \ \varphi=0 \text{ sobre }\partial\Omega.
\end{equation}
Pero esta identidad tiene sentido sin más que asumir que las funciones $f$ y $g$, así como sus derivadas, sean funciones de $L^2(\Omega)$. Utilizaremos este hecho para generalizar el concepto de derivada.

\step 
Definimos el \deff{soporte} de una función $\varphi:\Omega\to\R $ como el conjunto
\[
  \mathop{\rm sop}(\varphi) = \overline{\{\xx\in\Omega \ /\ \varphi(\xx)\neq 0\}},
\]
donde $\overline A$ denota a la clausura de un conjunto $A$ cualquiera en $\R$. 

\step
Denotamos por $\TestSpace(\Omega)$ al espacio de funciones%
%
\footnote{
  Como ejercicio, se puede comprobar que la función $g(x)=\exp(1/(x^2-1))$ si $|x|\le 1$, $g(x)=0$ si $|x|> 1$, es de $C^\infty(\R)$ y $\mathop{\rm sop}(g) = [-1,1]$. Por tanto $g\in\TestSpace(\R)$ y $g\in\TestSpace(\Omega)$ para todo $\Omega=(a,b)\subset\R$ tal que $[-1,1]\subset(a,b)$.  Pueden consultarse más ejemplos de funciones de $\TestSpace(\Omega)$,  por ejemplo, en~\cite{adams2003sobolev,sayas2019variational}.
}
%
infinitamente derivables cuyo soporte es un compacto contenido en $\Omega$. 
\[
  \TestSpace(\Omega) = \big\{\varphi\in C^\infty(\Omega) \ /\ \mathop{\rm sop}(\varphi) \text{ es compacto y }\mathop{\rm sop}(\varphi)\subset\Omega \big\}.
\]
No es difícl demostrar que toda función de $\TestSpace(\Omega)$ es cero «en las cercanías%
%
% \footnote{Para cualquier punto $\xx_0\in\partial\Omega$, existe una bola $B_{\xx_0}$ tal que $\varphi\equiv 0$ sobre $B_{\xx_0}\cap\Omega.$}
\footnote{Si $\varphi\in \TestSpace(\Omega)$, entonces existe un abierto $\Omega_0$ tal que $\mathop{\rm sop}(\varphi)\subset\overline\Omega_0\subset\Omega$ y por tanto $\varphi|_{\partial\Omega_0}=0$.}
%
de $\partial\Omega$. Ahora estamos en disposición de dar sentido a una fórmula de integración por partes similar a~\eqref{eq:partes.3} que usaremos para generalizar el concepto de derivada.

\step
Decimos que una función $u\in L^2(\Omega)$ es \deff{derivable en sentido débil}  (en $L^2(\Omega)$ y con respecto a la variable $x_i$) si existe alguna función $v\in L^2(\Omega)$ tal que
\[
  \int_\Omega v\, \varphi\; = -\int_\Omega u \, \partial_{x_i} \varphi \quad \forall \varphi \in \TestSpace(\Omega).
\]
En este caso, decimos que $v$ es la \deff{derivada débil} de $u$ con respecto a $x_i$ y la denotamos del mismo modo que a la derivada convencional: $v=\partial_{x_i}u$ o $v=u_{x_i}$ o $v=\partial u/\partial{x_i}$ o $v=D_i u$. 

Operadores usuales como el vector \deff{gradiente} o el operador \deff{divergencia} se generalizan a derivadas débiles. La idea de derivada débil se generaliza al concepto más elegante de de distribuciones~\cite{adams2003sobolev}.

\step[\textbf{Propiedades} de la derivada débil.]%
%
\footnote{Parea más detalles, demostaciones, etc, ver, por ejemplo,~\cite{adams2003sobolev}.}
%
\begin{enumerate}
  \item La derivada débil es única salvo conjuntos de medida nula. Esto es, si existe la derivada débil de $u\in L^2(\Omega)$ con respecto  $x_i$, ésta es la única función $u_{x_i}\in L^2(\Omega)$ tal que: 
\[
  \int_\Omega u_{x_i} \, \varphi\; = -\int_\Omega u\, \varphi_{x_i} \quad \forall \varphi \in \TestSpace(\Omega).
\]
\item Si una función es derivable en sentido clásico, entonces existe su derivada débil y coincide (c.p.d.) con la clásica. Esta es una consecuencia directa de~\eqref{eq:partes.3}. 
  \item  Las reglas clásicas para las derivadas de sumas y productos de funciones también son válidas para la derivada débil.
\end{enumerate}

\step
Existen funciones que son derivables en sentido débil pero no en sentido clásico. Por ejemplo, la función $u(x)=|x|$ en $\Omega=(-1,1)\subset\R$ es derivable en sentido débil, siendo\footnote{Se recomienda comprobarlo como ejercicio.} 
    \[
      \partial_x u =u'(x)=  \left\{\begin{aligned}-1 & \quad\text{si } x\le 0,\\ 1 &\quad\text{si } x>0. \end{aligned}\right.
    \]

    Nota: se puede demostrar que esta última función, $u'(x)$, no es derivable en sentido débil\footnote{En el caso $d=1$, todas las funciones con derivada débil en $L^2(\Omega)$ son continuas.}. Por tanto no existe una «derivada segunda en sentido débil» de $u(x)=|x|$. 

\step
Definimos el \deff{espacio de Sobolev} $H^1(\Omega)$ como el conjunto de las funciones que tienen todas sus derivadas débiles en $L^2(\Omega)$, esto es
\[
  H^1(\Omega) = \big\{v\in L^2(\Omega) \ /\ \grad v \in [L^2(\Omega)]^d \big\}.
\]

\step 
$H^1(\Omega)$ es un espacio de Hilbert (y por tanto un espacio norrmado completo) para el producto escalar
\[
  (u,v)_{H^1(\Omega)} = (u,v)_{L^2(\Omega)} + (\grad u, \grad v)_{L^2(\Omega)},
\]
cuya norma subordinada es
\[
  \norm[H^1(\Omega)]{v} = \left(\int_{L^2(\Omega)}v^2 +
  \int_{L^2(\Omega)} |\grad v|^2\right)^{1/2}.
\]
\step Puesto que las funciones de $C^1(\overline\Omega)$ y sus derivadas son de $L^2(\Omega)$, se tiene que   
\[
C^1(\overline\Omega) \subset H^1(\Omega).
\]

\step 
Consideremos la siguiente función, a la que llamamos \deff{función traza} sobre% 
\footnote{El concepto de traza se puede extender a otras regiones de la frontera, $\gamma:C^0(\Omega) \to C^0(\Gamma)$, con $\Gamma\subset\partial\Omega$.}
%
$\partial\Omega$:
\[
  \gamma:C^0(\overline\Omega) \to C^0(\partial\Omega),
\]
que asocia a cada función de $u\in C^0(\overline\Omega)$ su restricción a $\partial\Omega$, $\gamma(u)=u|_{\partial\Omega}$.

El siguiente resultado, una de las claves en la teoría de espacios de Sobolev, garantiza que se puede definir también una la restricción a la frontera%
%
\footnote{Obsérvese que la frontera es un conjunto de medida nula, por tanto no tendría sentido el definir la restricción a $\partial\Omega$ de funciones medibles Lebesgue (y por tanto definida c.p.t. $\xx\in\Omega$).\par 
Por otra parte, es necesario dotar de sentido preciso al espacio $L^2(\partial\Omega)$, ver por ejemplo~\cite{sayas2019variational}.}
%%
de funciones de $H^1(\Omega)$. 

\step{\textbf{Teorema.}}\footnote{Ver por ejemplo~\cite{sayas2019variational} para una demostración.} 
Existe una función lineal y continua, a la que seguiremos llamando traza y denotando por $\gamma$,
\[
  \gamma:  H^1(\Omega) \to L^2(\partial\Omega),
\]
que para funciones continuas coincide con la restricción a $\partial\Omega$, es decir verifica:
\[
  \gamma(u) = u|_{\partial\Omega} \quad  \forall u\in C^0(\overline\Omega)\cap H^1(\Omega).
\]
Abusando del lenguaje, diremos que $u|_{\partial\Omega}=g$ si $\gamma(u)=g$.

\step
En el caso de condiciones de contorno Dirichlet (homogéneas y no homogéneas), serán muy importantes las funciones que \textit{son cero en la frontera}. Denotamos por $H_0^1(\Omega)$ al núcleo de $\gamma$. Es decir:
\[
  H_0^1(\Omega) = \{ u\in H^1(\Omega) \ /\ \gamma(u)=0\}.
\]

\step{\textbf{Teorema.}}%
\footnote{El hecho de que los espacios de Sobolev pueden ser definidos por completitud no es evidente. Este fue el objeto de una curiosa publicación de sólo dos páginas con el intenioso título «$H=W$»~\cite{meyers1964h}.}
Se verifica la siguiente propiedad: $\TestSpace(\Omega)\subset H_0^1(\Omega)$ y de hecho%
$$H_0^1(\Omega) = \overline{\TestSpace(\Omega)}^{H^1(\Omega)},$$ clausura respecto a la norma de $H^1(\Omega)$.

\step{\textbf{Teorema (Desigualdad de Poincaré-Friedrichs).} Si $\Omega$ es acotado\footnote{Basta acotado en alguna dirección}, existe una constante $C>0$ tal que 
\[
  \norm[\Omega]{u} \le C \norm[\Omega]{\grad u}, \quad\forall u\in H_0^1(\Omega).
\]

\step Consecuencia: la siguiente expresión define una norma en $H_0^1(\Omega)$, que es equivalente a la norma de $H^1(\Omega)$.
\[
  \norm[H_0^1]{u} = \left(\int_{\Omega} \norm{\grad u}^2 \right)^{1/2}.
\]
Esta es la norma subordinada al siguiente producto escalar:
\[
  (u,v)_{H^0_1(\Omega)} = (\grad u, \grad v)_{L^2(\Omega)} = \int_{\Omega} \grad u \cdot \grad v
\]

%----------------------------------------------------
\section{El Problema de Poisson--Dirichlet Unidimensional}
%----------------------------------------------------

Comenzaremos considerando particular de~\eqref{eq:problema.modelo} en el que $d=1$, siendo $\Omega=(a,b)\subset\R$, $a,b\in\R$, y con condiciones de contorno Dirichlet homogéneas. Es decir, $\Gamma_D=\{a,b\}$ y $\Gamma_N=\emptyset$. Tenemos así el siguiente problema de Poisson-Dirichlet: 
\begin{equation}
  \left\{
    \begin{aligned}
      - \uxx &= f \en{\Omega}, \\
      u&=0 \sobre{\Gamma_D}, \\
    \end{aligned}
  \right.
  \label{eq:problema.dirichlet.homogeneo.1d}
\end{equation}
siendo $\uxx = d^2 u/d x^2$ y $f\in C^0(\Omega)$. Una solución clásica será una función $u\in C^0(\overline\Omega)\cap C^2(\Omega)$ para las que se verifica~\eqref{eq:problema.dirichlet.homogeneo.1d} de forma puntual, es decir $-u_{xx}(x)=f(x)$ para todo $x\in(a,b)$ y $u(a)=u(b)=0$.


%----------------------------------------------------
\subsection{Formulación Variacional}
%----------------------------------------------------

En este apartado se estudia una formulación alternativa para el problema~\eqref{eq:problema.dirichlet.homogeneo.1d}. Veremos que ésta permite establecer condiciones suficientes para la existencia y unicidad de solución en un sentido débil,  más general que el de solución clásica. Estas soluciones débiles serán, de hecho, soluciones clásicas siempre que los datos sean suficientemente regulares.

Como veremos luego, este tipo de formulación débil o variacional es generalizable de forma sencilla a dimensión $d>1$, así como a  condiciones de contorno más generales. Y será la base sobre la que  se construye el método de los elementos finitos. 


\step
Supongamos que $u$ es una sjlución clásica del problema~\eqref{eq:problema.dirichlet.homogeneo.1d} y que $v$ es una función de $C^0(\overline\Omega)\cap C^1(\Omega)$ tal que $v=0$ sobe $\Gamma_D$. Entonces podemos aplicar la fórmula de Green de integración por partes~\eqref{eq:green}, obteniendo:
\begin{equation*}
  \int_\Omega (-\uxx)\,v  = -  \int_a^b \uxx(x) v(x) \,dx = 
  \int_a^b u_x(x) v_x(x)\,dx - \big( u(b)v(b) - u(a)v(a)  \big).
\end{equation*}
Usando que $v(a)=v(b)=0$ y que $-\uxx(x)=f(x)$ para todo $x\in(a,b)$:
\[
  \int_a^b u_x(x) v_x(x)\,dx  = \int_a^b f(x) v(x) \, dx.
\]
Esta expresión la podemos denotar como
\[
  a(u,v)=L(v),
\]
para
\begin{align}
  a(u,v)  &= \int_a^b u_x(x) v_x(x)\,dx,  
  \label{eq:forma.bilineal.a}
\\
  L(v)  &= \int_a^b f(x) v(x) \, dx.
  \label{eq:forma.lineal.L}
\end{align}


Así, hemos visto que toda solución clásica $u$ de~\eqref{eq:problema.dirichlet.homogeneo.1d} es en particular solución del siguiente problema sobre el espacio  $V^* = \{ v\in C^0(\overline\Omega)\cap C^1(\Omega) \ /\ v\in V^*,\  v(a)=v(b)=0\}$:
\begin{equation}
  \label{eq:problema.variacional.C1}
\begin{aligned}
  \text{Hallar }u\in V^*  \enspace\text{tal que:} \enspace
a(u,v) = L(v), \quad \forall v\in V^*.
\end{aligned}
\end{equation}

\step Sin embargo, a pesar de que el problema~\eqref{eq:problema.variacional.C1} es más débil que el original (pues la solución sólo requiere tener derivadas primeras), no podremos garantizar existencia y unicidad de solución, esencialmente debido a que el espacio $V^*$ no es completo para una norma adecuada%
%
\footnote{
  Intuitivamente, podemos considerar $D(\Omega)\subset V^*$ pero recordemos que $\overline {D(\Omega)}^{H^1(\Omega)} = H_0^1(\Omega)$, un espacio más amplio que $V^*$.
} 
% 
.
De hecho, para espacios espacios de Hilbert (es decir, completos para la norma asociada a su producto escalar) tenemos un resultado muy potente que introduciremos en la siguiente proposición.

\step 
Decimos que una forma bilineal sobre $V\times V$ es \deff{elíptica} o \deff{coercitiva} si existe una constante $\alpha>0$ tal que
\[
  a(v,v) \ge \alpha \norm[V]{v}^2\quad\forall v\in V.
\]

\step{\textbf{Proposición (Lema de Lax-Milgram)}}\footnote{Puede consultarse una demostración en~\cite{adams2003sobolev, sayas2019variational}.}
Sea $V$ un espacio de Hilbert y $a:V\times V\to\R$ una forma bilineal, continua y coercitiva. Entonces, para toda $L:V\to\R$ lineal y continua existe una única $u\in V$  tal que
\[
  a(u,v) = L(v) \quad\forall v\in V. 
\]
Además, esta solución verifica la siguiente estimación de tipo estabilidad\footnote{Cuya demostración es sencilla, se deja como ejercicio}:
\[
  \norm[V]{u}\le \frac 1 {\alpha}\;\sup_{0\neq v\in V}\frac{|a(u,v)|}{\norm[V]{v}}.
\]

\step 
Nos proponemos, pues, escribir una formulación débil del problema~\eqref{eq:problema.dirichlet.homogeneo.1d} que se encuentre en las hipótesis del Lema de Lax-Milgram. Para ello basta observar que las aplicaciones $a(\cdot,\cdot)$ y $F(\cdot)$, definidas en~\eqref{eq:forma.bilineal.a} y~\eqref{eq:forma.lineal.L} tengan sentido es suficiente que $u,v\in H_0^1(\Omega)$ y que $f\in L^2(\Omega)$. 

\step
Planteamos, por tanto, el siguiente problema, al que llamamos \deff{problema variacional} asociado a~\eqref{eq:problema.dirichlet.homogeneo.1d}:
\begin{equation}
  \label{eq:problema.variacional.1d}
  \text{Hallar }u\in V=H_0^1(\Omega)  \enspace\text{tal que} \enspace
  a(u,v) = F(v) \quad \forall v\in H_0^1(\Omega),
\end{equation}
con $a(\cdot,\cdot)$ y $F(\cdot)$, definidas en~\eqref{eq:forma.bilineal.a} y~\eqref{eq:forma.lineal.L}.


\step{\textbf{Proposición}.} Para todo $f\in L^2(\Omega)$, existe una única solución del problema variacional~\eqref{eq:problema.variacional.1d}, que verifica la siguiente acotación de estabilidad:
\begin{equation}
  \label{eq:Lax-Milgram.estimación}
  \norm[H_0^1]{u}\le \frac C {\alpha}\; \norm[L^2(\Omega)]{f},
\end{equation}
donde  $\alpha>0$ y $C>0$ son, respectivamente, las constantes de coercitividad y de la desigualdad de Poincaré-Friedrichs.

\begin{proof}
  Idea:
  \begin{enumerate}
    \item Consideremos el espacio $V=H_0^1(\Omega)$, que es un espacio de Hilbert para el producto escalar $(\cdot,\cdot)_{H^1(\Omega)}$, pues es un subespacio cerrado del espacio de Hilbert $H^1(\Omega)$. 
    \item $a(\cdot,\cdot)$ y $L(\cdot)$ son respectivamente, una forma bilineal continua y una alicación lineal continua sobre $V$. Esto es evidente usando la desigualdad de Cauchy-Schwarz y teniendo en cuenta la norma $\norm[H_0^1(\Omega)]{\cdot}$ (que es equivalente a la $\norm[H^1(\Omega)]{\cdot}$.
    \item La forma bilineal $a(\cdot,\cdot)$ es coercitiva. De hecho, por definición, $a(v,v)=\norm[H_0^1(\Omega)]{v}^2$ para todo $v\in V$. Podemos tomar la constante de coercitividad $\alpha=1$.
  \item Por tanto podemos aplicar Lema de Lax-Milgram, de donde existe una única solución de~\eqref{eq:problema.variacional.1d}.Aplicaremos~\eqref{eq:Lax-Milgram.estimación} y, consecutivamente: (1) $a(u,v)=(f,v)_{L^2(\Omega)}$ para todo $v\in V$, (2) la desigualdad de Cauchy-Schwarz y (3) la desigualdad de Poincaré-Friedrichs (con constante $C>0$):
      \begin{align*}
        \norm[V]{u} \le 
        \frac 1 {\alpha}\;\sup_{0\neq v\in V}\frac{|a(u,v)|}{\norm[V]{v}} =
        \frac 1 {\alpha}\;\sup_{0\neq v\in V}\frac{|(f,v)_{L^2(\Omega)}|}{\norm[V]{v}}
        \le
        \frac 1 {\alpha}\;\sup_{0\neq v\in V}\frac{\norm[L^2(\Omega)]{f}\norm[L^2(\Omega)]{v}}{\norm[V]{v}}
        \\
        \le
        \frac C {\alpha}\;\sup_{0\neq v\in V}\frac{\norm[L^2(\Omega)]{f}\norm[V]{v}}{\norm[V]{v}}
        =
        \frac C {\alpha}\;\norm[L^2(\Omega)]{f}.
      \end{align*}
  \end{enumerate}
\end{proof}

\step{\textbf{Proposición}.}
Si los datos son suficiente regulares ($f\in C^0(\Omega)$), entonces la solución débil es una solución clásica del problema~\eqref{eq:problema.dirichlet.homogeneo.1d}.

\section{El Método de los Elementos Finitos}

El método de los elementos finitos para el problema de Poisson-Dirichlet 1-dimensional~\eqref{eq:problema.dirichlet.homogeneo.1d} puede expresarse como: dada una partición $a=x_0<x_1<\cdots<x_n=b$ del intervalo $\overline\Omega=[a,b]$, aproximar el espacio $V=H_0^1(\Omega)$ por el subespacio vectorial
\[
  V_h = \left\{ v_h\in C^0(\overline\Omega) \ /\ v|_{[x_{i-1},x_i]}\in\P_m \quad \forall i=1,\dots,n, \quad v(a)=v(b)=0 \right\},
\]
siendo $\P_m$ el espacio de los polinomios de grado $m\in\N$.

En concreto, sobre este espacio $V_h\subset V$ planteamos el problema:
\begin{equation}
  \label{eq:problema.elementos-finitos.1d}
  \text{Hallar }u_h\in V_h  \enspace\text{tal que} \enspace
  a(u_h,v_h) = F(v_h) \quad \forall v_h\in V_h.
\end{equation}

\bibliographystyle{unsrt}
\bibliography{biblio}
\end{document}


